%template1.tex
%The following LaTeX source file represents the simplest kind of slide presentation; no overlays, no included graphics. Substitute your favorite style for ``pascal''. To create the PDF file template1.pdf, (1) be sure to use the prosper class, then (2) execute the command latex template1.tex, and (3) the command dvipdf template1.dvi.

%%%%%%%%%%%%%%%%%%%%%%%%%%%%%%% template1.tex %%%%%%%%%%%%%%%%%%%%%%%%%%%%%%%%%%%
\documentclass[a4paper,blends,pdf,colorBG,slideColor]{prosper}
% definitions for slides for CSC544
% Lutz Hamel, (c) 2007

\hypersetup{pdfpagemode=FullScreen}

\usepackage{times}
\usepackage{latexsym}
\usepackage{alltt}
\usepackage{booktabs}
\usepackage{amsmath}
\usepackage{amsopn}
\usepackage{amsfonts}
\usepackage{amssymb}
%\usepackage[usenames]{color}

\def\sign{\qopname\relax{no}{sign}}
\def\argmax{\qopname\relax{no}{argmax}}
\def\argmin{\qopname\relax{no}{argmin}}

\newcommand{\grad}{\ensuremath{\nabla}} 
\newcommand{\loss}{\ensuremath{{\cal L}}}
\newcommand{\err}{\mbox{err}}
\newcommand{\mse}{\mbox{mse}}
\newcommand{\acc}{\mbox{acc}}
\newcommand{\Integer}{\ensuremath{\mathbb{N}}}
\newcommand{\size}[1]{{|{#1}|}}
\newcommand{\Rnspace}[1]{\ensuremath{\mathbb{R}^{#1}}}
\newcommand{\Real}{\ensuremath{\mathbb{R}}}
\newcommand{\mytt}[1]{{\small\tt{#1}}}
\newcommand{\textemph}[1]{{\em #1}}
\newcommand{\suchthat}{\mid}
\newcommand{\orbar}{\;|\;}
\newcommand{\bs}[1]{\begin{slide}{#1}\ptsize{8}}
\newcommand{\es}{\end{slide}}
\newcommand{\co}{\,\colon\;}
\newcommand{\pair}[2]{\ensuremath{( {#1}, {#2} )}}
\newcommand{\model}[1]{\hat{#1}}
\newcommand{\ul}[1]{{\bf\em #1}}
\newcommand{\ol}{\overline}
\newcommand{\definition}[1]{{\bf Definition: }{\em #1}}
\newcommand{\example}[1]{{\bf Example: }{#1}}
\newcommand{\abs}[1]{|{#1}|}
\newcommand{\mytab}{\makebox[.1in]{}}

\newcommand{\fdef}[1]{
\begin{center}
\fbox{
\begin{minipage}{3.5in}
{\bf Definition:}
{#1}
\end{minipage}
}
\end{center}
}

\newcommand{\fframe}[1]{
\begin{center}
\fbox{
\begin{minipage}{3.5in}
{#1}
\end{minipage}
}
\end{center}
}

\newcommand{\nframe}[1]{
\begin{center}
\begin{minipage}{3.5in}
{#1}
\end{minipage}
\end{center}
}

\newenvironment{Rcode}
	{
		\scriptsize
		\begin{quote}
		\begin{alltt}
	}
	{
		\end{alltt}
		\end{quote}
	}




\begin{document}

\bs{Kernel Functions}
\small
If we let $k(\ol{x},\ol{y}) =\Phi(\ol{x})\bullet\Phi(\ol{y}) $ be a kernel function, then we can write
our support vector machine in terms of kernels,
\begin{equation*}
\model{f}(\ol{x}) =  \sign\left (\sum_{i=1}^l \alpha^*_i y_i  {\color{red}k(\ol{x}_i,\ol{x})}  -  \sum_{i=1}^l \alpha^*_i y_i  {\color{red}k(\ol{x}_i,\ol{x}_{sv^+})} + 1\right )
\end{equation*}

We can write our training algorithm in terms of kernel functions as well,
\begin{equation*}
 \ol{\alpha}^* = \argmax_{\ol{\alpha}} \left ( \sum_{i=1}^l \alpha_i - 
  \frac{1}{2}\sum_{i=1}^l \sum_{j=1}^l \alpha_i \alpha_j y_i y_j {\color{red}k(\ol{x}_i,\ol{x}_j)} \right ),
\end{equation*}  
subject to the constraints,
\begin{align*}
\sum_{i=1}^l \alpha_i  y_i &= 0,\\
\alpha_i  &\ge 0, \quad i = 1,\ldots,l.
\end{align*}
Selecting the right kernel for a particular non-linear classification problem is called {\em feature search}.
\es

\bs{Kernel Functions}
\begin{tabular*}{\textwidth}{@{\vrule height 16pt depth 4pt width 0pt \hskip\arraycolsep\extracolsep{\fill}}l l l}
      \toprule
Kernel Name & Kernel Function & Free Parameters \\
      \midrule
Linear Kernel & 
	$k(\ol{x},\ol{y}) = \ol{x} \bullet \ol{y} $ & 
		none \\ 
Homogeneous Polynomial Kernel & 
	$k(\ol{x},\ol{y}) = (\ol{x} \bullet \ol{y})^d$ & 
		$d \geq 2$ \\ 
Non-Homogeneous Polynomial Kernel & 
	$k(\ol{x},\ol{y}) = (\ol{x} \bullet \ol{y} + c)^d$ & 
		$d \geq 2$, $c > 0$ \\ 
Gaussian Kernel & 
	$k(\ol{x},\ol{y}) = e^{-\frac{\abs{\ol{x}-\ol{y}}^2}{2\sigma^2}}$ & 
		$\sigma > 0$ \\ 
      \bottomrule
   \end{tabular*}
\es

\bs{Non-linear Classifiers}
Let's review classification with non-linear SVMs:
\begin{enumerate}
\item We have a non-linear data set.
\item Pick a kernel other than the linear kernel, this means that the input space will be transformed
into a higher dimensional feature space.
\item Solve our dual maximum margin problem in the feature space (we are solving now a linear 
classification problem).
\item The resulting model is a linear model in feature space and a {\em non-linear model} in
input space.
\end{enumerate}
\es

\bs{A Closer Look at Kernels}

We have shown that for $\Phi(x_1,x_2) = (x_1^2,x_2^2,\sqrt{2}x_1 x_2)$ the kernel
\begin{equation*}
k(\ol{x},\ol{y}) = \Phi(\ol{x})\bullet\Phi(\ol{y}) = (\ol{x}\bullet\ol{y})^2.
\end{equation*}
That is, we picked our mapping from input space into feature space in such a way that the kernel
in feature space can be evaluated in input space.

This begs the question: What about the other kernels? What do the associated feature spaces
and mappings look like?

It turns out that for each kernel function we can construct a canonical feature space and mapping.
This implies that features spaces and mappings for kernels are not unique!
\es

\bs{Properties of Kernels}
\small
\vspace{.2in}
\fdef{[Positive Definite Kernel]
A function $k\co \Rnspace{n} \times \Rnspace{n} \rightarrow \Real$ such that
\begin{equation*}
\sum_{i=1}^l\sum_{j=1}^l\theta_i \theta_j k(\ol{x}_i,\ol{x}_j) \geq 0 
\end{equation*}
holds is called a \ul{positive definite kernel}.
Here, $\theta_i,\theta_j \in \Real$ and  
$\ol{x}_1,\ldots,\ol{x}_l$ is a set of points in $\Rnspace{n}$.
}
\es

\bs{Properties of Kernels}
\small
New notation:  Let $k\co \Rnspace{n} \times \Rnspace{n} \rightarrow \Real$
be a kernel, then $k(\cdot, \ol{x})$ is a partially evaluated kernel with $\ol{x} \in \Rnspace{n}$ and
represents a function $\Rnspace{n} \rightarrow \Real$.

\vspace{.2in}
\fframe{{\bf Theorem:} [Reproducing Kernel Property]
\label{th:reproducible-kernel}
Let $k\co \Rnspace{n} \times \Rnspace{n} \rightarrow \Rnspace{n}$ be a positive definite kernel, then
the following property holds,
\begin{equation*}
k(\ol{x},\ol{y}) = k(\ol{x},\cdot)\bullet k(\cdot,\ol{y}),
\end{equation*}
with $\ol{x},\ol{y} \in \Rnspace{n}$.
}
\es

\bs{Feature Spaces are not Unique}
\small
We illustrate that feature spaces are not unique using our homogeneous polynomial kernel to the power of two, that is, 
$k(\ol{x},\ol{y})=(\ol{x}\bullet\ol{y})^2$ with $\ol{x}, \ol{y} \in \Rnspace{2}$.
Let $\Phi\co \Rnspace{2} \rightarrow {\color{red}\Rnspace{3}}$ such that
\begin{equation*}
\Phi(\ol{x}) = \Phi(x_1,x_2)= ({x}_1^2,{x}_2^2,\sqrt{2} {x}_1^2 {x}_2^2)
\end{equation*}
and $\Psi\co \Rnspace{2} \rightarrow {\color{red}\{\Rnspace{2}\rightarrow \Real\}}$ with
\begin{equation*}
\Psi(\ol{x}) = k(\cdot,\ol{x}) = ((\cdot)\bullet\ol{x})^2,
\end{equation*}
be two mappings from our input space to two different feature spaces,
then
\begin{align*}
\Phi(\ol{x})\bullet\Phi(\ol{y}) &= (\ol{x}_1^2,\ol{x}_2^2,\sqrt{2} \ol{x}_1^2 \ol{x}_2^2)\bullet
		(\ol{y}_1^2,\ol{y}_2^2,\sqrt{2} \ol{y}_1^2 \ol{y}_2^2)\\
		& =  (\ol{x}\bullet \ol{y})^2\\
		&= k(\ol{x},\ol{y})\\
		&= k(\cdot,\ol{x})\bullet k(\cdot,\ol{y})\\
		& =  ((\cdot)\bullet\ol{x})^2 \bullet ((\cdot)\bullet\ol{y})^2\\
		& =  \Psi(\ol{x}) \bullet \Psi(\ol{y}).
\end{align*}
The section on kernels in the book shows that the construction $\Psi(\ol{x}) \bullet \Psi(\ol{y})$
is indeed well defined and represents a dot product in an appropriate feature space.
\es

\end{document}
%%%%%%%%%%%%%%%%%%%%%%%%%%% end of template1.tex %%%%%%%%%%%%%%%%%%%%%%%%%%%%%%%%

