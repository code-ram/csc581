%template1.tex
%The following LaTeX source file represents the simplest kind of slide presentation; no overlays, no included graphics. Substitute your favorite style for ``pascal''. To create the PDF file template1.pdf, (1) be sure to use the prosper class, then (2) execute the command latex template1.tex, and (3) the command dvipdf template1.dvi.

%%%%%%%%%%%%%%%%%%%%%%%%%%%%%%% template1.tex %%%%%%%%%%%%%%%%%%%%%%%%%%%%%%%%%%%
\documentclass[a4paper,blends,pdf,colorBG,slideColor]{prosper}
% definitions for slides for CSC544
% Lutz Hamel, (c) 2007

\hypersetup{pdfpagemode=FullScreen}

\usepackage{times}
\usepackage{latexsym}
\usepackage{alltt}
\usepackage{booktabs}
\usepackage{amsmath}
\usepackage{amsopn}
\usepackage{amsfonts}
\usepackage{amssymb}
%\usepackage[usenames]{color}

\def\sign{\qopname\relax{no}{sign}}
\def\argmax{\qopname\relax{no}{argmax}}
\def\argmin{\qopname\relax{no}{argmin}}

\newcommand{\grad}{\ensuremath{\nabla}} 
\newcommand{\loss}{\ensuremath{{\cal L}}}
\newcommand{\err}{\mbox{err}}
\newcommand{\mse}{\mbox{mse}}
\newcommand{\acc}{\mbox{acc}}
\newcommand{\Integer}{\ensuremath{\mathbb{N}}}
\newcommand{\size}[1]{{|{#1}|}}
\newcommand{\Rnspace}[1]{\ensuremath{\mathbb{R}^{#1}}}
\newcommand{\Real}{\ensuremath{\mathbb{R}}}
\newcommand{\mytt}[1]{{\small\tt{#1}}}
\newcommand{\textemph}[1]{{\em #1}}
\newcommand{\suchthat}{\mid}
\newcommand{\orbar}{\;|\;}
\newcommand{\bs}[1]{\begin{slide}{#1}\ptsize{8}}
\newcommand{\es}{\end{slide}}
\newcommand{\co}{\,\colon\;}
\newcommand{\pair}[2]{\ensuremath{( {#1}, {#2} )}}
\newcommand{\model}[1]{\hat{#1}}
\newcommand{\ul}[1]{{\bf\em #1}}
\newcommand{\ol}{\overline}
\newcommand{\definition}[1]{{\bf Definition: }{\em #1}}
\newcommand{\example}[1]{{\bf Example: }{#1}}
\newcommand{\abs}[1]{|{#1}|}
\newcommand{\mytab}{\makebox[.1in]{}}

\newcommand{\fdef}[1]{
\begin{center}
\fbox{
\begin{minipage}{3.5in}
{\bf Definition:}
{#1}
\end{minipage}
}
\end{center}
}

\newcommand{\fframe}[1]{
\begin{center}
\fbox{
\begin{minipage}{3.5in}
{#1}
\end{minipage}
}
\end{center}
}

\newcommand{\nframe}[1]{
\begin{center}
\begin{minipage}{3.5in}
{#1}
\end{minipage}
\end{center}
}

\newenvironment{Rcode}
	{
		\scriptsize
		\begin{quote}
		\begin{alltt}
	}
	{
		\end{alltt}
		\end{quote}
	}




\begin{document}

\bs{Optimization Problem}
\small
Recall the following setting for training support vector machines.

Assume that we are given the training set,
\begin{equation*}
D = \{(\ol{x}_1,y_1),(\ol{x}_2,y_2),\ldots,(\ol{x}_l,y_l)\} \subseteq \Rnspace{n}\times\{+1, -1\}.
\end{equation*}
We are interested in computing a classifier in the form of a support vector machine model,
\begin{equation*}
\model{f}(\ol{x}) = \sign \left ( \sum_{i=1}^l y_i \alpha^*_i k(\ol{x}_i,\ol{x}) - b^*\right ),
\end{equation*}
using a training algorithm based on the Lagrangian dual,
\begin{equation*}
\ol{\alpha}^* = {\color{red} \argmax_{\ol{\alpha}} \phi'(\ol{\alpha})} = 
 \argmax_{\ol{\alpha}}  \left (\sum_{i=1}^l \alpha_i - 
  \frac{1}{2}\sum_{i=1}^l\sum_{j=1}^l y_i y_j \alpha_i \alpha_j k(\ol{x}_i,\ol{x}_j) \right ),
\end{equation*}
subject to  the constraints,
\begin{align*}
\sum_{i=1}^l y_i \alpha_i &= 0, \\
C \ge \alpha_i &\ge 0,  
\end{align*}
with $i=1,\ldots,l$.
\es

\bs{Quadratic Programming}
We can use quadratic programming to implement support vector machines,
\begin{gather*}
\ol{\alpha}^* = \argmin_{\ol{\alpha}} \left (\frac{1}{2} \ol{\alpha}^T {\mathbf Q}\ol{\alpha} - \ol{q}\bullet\ol{\alpha}  \right ),\\
\intertext{subject to the constraints,}
{\color{red}{\mathbf Y} \ol{\alpha} = 0},\\
{\color{red}\ol{u} \le \ol{\alpha} \le \ol{v}}.
\end{gather*}
In order to do this we have to use quadratic programming packages
that support both equality and inequality constraints.

This is different from our use of quadratic programming in the primal 
maximum margin classifiers - there we only needed to support inequality constraints,
\begin{equation*}
\ol{w}^* =  \argmin_{\ol{w}} \left (\frac{1}{2} \ol{w}^T {\mathbf Q}\; \ol{w} - \ol{q}\bullet\ol{w}\right ),
\end{equation*}
subject to the constraints
\begin{equation*}
{\mathbf X}^T\ol{w} \ge \ol{c}.
\end{equation*}

\es

\bs{Quadratic Programming}
\small
Quadratic solvers are usually defined in terms of minimization, therefore we need to
massage our support vector optimization problem a little bit,
\begin{align*}
\argmax_{\ol{\alpha}}  \phi'(\ol{\alpha}) &= \argmin_{\ol{\alpha}} \left (- \phi'(\ol{\alpha})\right ) \nonumber\\
\label{eq:lagrangian-dual-impl-min}
	&= \argmin_{\ol{\alpha}}  
	 \left (\frac{1}{2}\sum_{i=1}^l\sum_{j=1}^l y_i y_j \alpha_i \alpha_j k(\ol{x}_i,\ol{x}_j) 
 		 - \sum_{i=1}^l \alpha_i \right )\\
	&= \argmin_{\ol{\alpha}} \left (\frac{1}{2} \ol{\alpha}^T {\mathbf Q}\ol{\alpha} - \ol{q}\bullet\ol{\alpha}  \right )
\end{align*}
where the matrix $\mathbf Q$ is an $l\times l$ matrix with components 
\begin{equation*}
Q_{ij} = y_i y_j {\color{red}k(\ol{x}_i,\ol{x}_j)}
\end{equation*}
and the vector $\ol{q}$ has $l$ components initialized to one,
\begin{equation*}
\ol{q} = \ol{1}.
\end{equation*}
We call the matrix $\mathbf Q$ the {\em kernel matrix}. In effect, when using quadratic programming
solvers we need to {\em precompute} the dot product between the pairwise training points.
\es

\bs{Quadratic Programming}
\small
The constraints 
\begin{gather*}
{\mathbf Y} \ol{\alpha} = 0,\\
\ol{u} \le \ol{\alpha} \le \ol{v}.
\end{gather*}
are easily instantiated. We let $\mathbf Y$ be a $1 \times l$ matrix with components
\begin{equation*}
Y_{1i} = y_i,
\end{equation*}
where $(\ol{x}_i,y_i) \in D$ then,
\begin{equation*}
{\mathbf Y} \ol{\alpha} = \sum_{i = 1}^l y_i \alpha_i.
\end{equation*}
Finally, if we let $\ol{u} = \ol{0}$  and  let
vector $\ol{v}$ be a constant vector with $v_i = C$ then,
\begin{equation*}
u_i \le \alpha_i \le v_i,
\end{equation*}
which implies
\begin{equation*}
0 \le \alpha_i \le C,
\end{equation*}
with $i = 1,\ldots,l$.

\es

\bs{Quadratic Programming}
\fframe{
{\bf let} $D = \{(\ol{x}_1,y_1), (\ol{x}_2,y_2),\dots,(\ol{x}_l,y_l)\} \subset \Rnspace{n} \times \{+1,-1\}$\\
{\bf let} $C > 0$\\
{\bf let} {\rm $\mathbf Q$ be a $l\times l$ matrix with components $Q_{ij} = y_i y_j k(\ol{x}_i,\ol{x}_j)$}\\
{\bf let} {\rm  $\mathbf Y$ be a $1\times l$ matrix with components $Y_{1j} = y_j$}\\
{\bf let} {\rm $\ol{q}$ be a constant vector with $q_i = 1$}\\
{\bf let} {\rm $\ol{u}$ be a constant vector with $u_i = 0$}\\
{\bf let} {\rm $\ol{v}$ be a constant vector with $v_i = C$}\\
$\ol{\alpha} \leftarrow \text{solve}({\mathbf Q},\ol{q},{\mathbf Y},\ol{u},\ol{v})$\\
$b \leftarrow \text{\rm compute using known support vectors}$\\
{\bf return} $(\ol{\alpha},b)$
}
\begin{center}
Using a quadratic programming solver to train support vector machines.
\end{center}
\es

\bs{Quadratic Programming}
\vspace{.2in}
{\bf Observation:} Our kernel matrix $\mathbf Q$ is an $l\times l$ matrix on the training set $D$.
Consider a large training set with 50,000 observations.  This would mean that the kernel
matrix contains 2.5 billion elements, or consumes 10GB of memory if we represent each element with a 4 byte word.

$\Rightarrow$ Not feasible on most of today's architectures.
\es

\bs{Chunking}
\vspace{.2in}
{\bf Idea:} For most data sets the ratio of support vectors to training instances is very low; we say
that solving the dual optimization problem for SVMs gives rise to a {\em sparse solution}.

The idea is then to break up the training data into {\em chunks} and search for support vectors in
the individual chunks.

If we keep the size of the chunks or {\em working sets} small then the original optimization
problem breaks down into smaller, manageable optimization sub-problems.
\es

\bs{Chunking}
\small
\fframe{
{\bf let} $D = \{(\ol{x}_1,y_1), (\ol{x}_2,y_2),\dots,(\ol{x}_l,y_l)\} \subset \Rnspace{n} \times \{+1,-1\}$\\
{\bf let} $k > 0$\\
$\ol{\alpha} \leftarrow \ol{0}$\\
{\rm select a subset $W$ of size $k$ from $D$}\\
{\bf repeat forever}\\
\mytab {\rm solve Lagrangian dual for $W$ (update $\ol{\alpha}$ accordingly)} \\
\mytab {\rm delete observations from $W$ that are not support vectors}\\
\mytab $b \leftarrow \text{\rm compute using support vectors in $W$}$\\
\mytab {\bf if} {\rm all $d\in D$ satisfy the KKT conditions } {\bf then}\\
\mytab\mytab {\bf return} $(\ol{\alpha},b)$\\
\mytab {\bf end if}\\
\mytab {\rm $D_k \leftarrow$ the $k$ worst offenders in $D$  of the KKT conditions}\\
\mytab $W \leftarrow {\color{red}W \cup D_k}$\\
{\bf end repeat} \\
}
\begin{center}
Solving the dual optimization problem using chunking.
\end{center}

\vspace{.2in}
{\bf Observation:} $\abs{W \cup D_k} \approx k$ when the solution is sparse.
\es

\bs{Chunking}
\small
We are trying to find a global maximum by solving smaller sub-problems - how do we know when 
we are done?

$\Rightarrow$ We monitor the KKT conditions,

\begin{align*}
\frac{\partial L}{\partial \ol{w}}(\ol{\alpha},\ol{\beta}, \ol{w}^*,\ol{\xi},b) &= 0,\\
\frac{\partial L}{\partial \xi_i}(\ol{\alpha},\ol{\beta},\ol{w},\xi^*_i,b) &= 0,\\
\frac{\partial L}{\partial b}(\ol{\alpha},\ol{\beta},\ol{w},\ol{\xi},b^*) &= 0,\\
\alpha^*_i (y_i(\ol{w}^*\bullet\ol{x}_i - b^*) + \xi^*_i -1)&= 0, \\
\beta^*_i\xi^*_i&= 0,\\
y_i(\ol{w}^*\bullet\ol{x}_i - b^*) + \xi^*_i -1&\ge 0, \\
\alpha^*_i &\ge 0,\\
\beta^*_i &\ge 0,\\
\xi^*_i &\ge 0,
\end{align*}
for $i = 1,\ldots,l$.
\es

\bs{Chunking}
\small
A closer look reveals that we only need to monitor the complimentarity conditions,
\begin{gather*}
\alpha_i (y_i(\ol{w}\bullet\ol{x}_i - b) + \xi_i -1)= 0, \\
\beta_i\xi_i = 0.
\end{gather*}
We can rewrite these slightly using the dual for the normal vector $\ol{w}$ and
$\beta_i = C - \alpha_i$ as,
\begin{gather*}
\alpha_i\left( y_i\left ( \sum_{j=1}^l y_j \alpha_j k(\ol{x}_j,\ol{x}_i) - b\right ) - 1 + \xi_i\right ) = 0,\\
(C - \alpha_i)\xi_i = 0,
\end{gather*}
for all $(\ol{x}_i,y_i)\in D$ with $i = 1,\ldots,l$.

\vspace{.2in}
{\bf Problem:} We cannot directly monitor the slack variables $\xi_i$.  However,
we can obtain a set of conditions implied by the complimentarity conditions using
a case analysis on $\alpha_i$.
\es

\bs{Chunking}
\vspace{.2in}
Case analysis:
\begin{description}
\item[$\alpha_i = 0$:] This implies $\xi_i = 0$, therefore
$ %\begin{equation*}
y_i\left (\sum_{j=1}^l y_j \alpha_j k(\ol{x}_j,\ol{x}_i) - b\right ) \ge 1.
$ %\end{equation*}
\item[$0 < \alpha_i < C$:] This implies $\xi_i = 0$, therefore
$ %\begin{equation*}
y_i\left (\sum_{j=1}^l y_j \alpha_j k(\ol{x}_j,\ol{x}_i) - b\right ) = 1.
$ %\end{equation*}
\item[$\alpha_i = C$:] This implies $\xi_i > 0$, therefore
$ %\begin{equation*}
y_i\left (\sum_{j=1}^l y_j \alpha_j k(\ol{x}_j,\ol{x}_i) - b\right ) \le 1.
$ %\end{equation*}
\end{description}
\es

\bs{Chunking}
\vspace{.2in}
Putting it all together gives us the condition:
\begin{equation*}
y_i\left (\sum_{j=1}^l y_j \alpha_j k(\ol{x}_j,\ol{x}_i) - b\right ) \left \{ 
\begin{aligned}
\ge 1 & \text{ if $\alpha_i = 0$}\\
= 1 & \text{ if $0 < \alpha_i < C$}\\
\le 1 & \text{ if $\alpha_i = C$}
\end{aligned}
\right .
\end{equation*}
for all $(\ol{x}_i,y_i)\in D$ with $i = 1,\ldots,l$.

{\bf Observation:} One way to view these conditions is as a {\em consistency test}: if the values of the Lagrangian multipliers
are consistent with the locations of the corresponding training points vis-\`{a}-vis a decision surface,
then we have found a global solution.

\es

\bs{SMO}

SMO -- Sequential Minimal Optimization.

We can view the SMO algorithm as a chunking algorithm with a working set of size 2.

A working set of size 2 is the smallest working set that allows for an optimization that
satisfies the constraints,
\begin{equation*}
\sum_{i=1}^l y_i \alpha_i = 0.
\end{equation*}
Then consider the  subproblem of our Lagrangian dual optimization using the Lagrangian multipliers $\alpha_j$
and $\alpha_k$ (our working set),
\begin{equation*}
\max_{\alpha_j,\alpha_k} \phi'(\alpha_j,\alpha_k),
\end{equation*}
subject to the constraints,
\begin{align*}
y_j \alpha_j + y_k \alpha_k &= \delta,\\
C \ge \alpha_j,\alpha_k &\ge 0,  
\end{align*}
with $j,k = 1, \ldots, l$ and $j\ne k$.

\es


\bs{SMO}
We can rewrite the second to last equation on the previous slide as,
\begin{equation*}
\alpha_k = y_k (\delta - y_j \alpha_j).
\end{equation*}
It follows that,
\begin{equation*}
\max_{\alpha_j,\alpha_k} \phi'(\alpha_j,\alpha_k) = \max_{\alpha_j} \phi' \left (\alpha_j,y_k(\delta - y_j\alpha_j)\right) = \max_{\alpha_j} \psi(\alpha_j).
\end{equation*}
That is, our two-variable optimization problem becomes an optimization problem over the single variable  $\alpha_j$.  We represent the objective function of this single variable optimization problem
as $\psi(\alpha_j)$.
Now $\psi$ has a single maximum that we can find by,
\begin{equation*}
\frac{d \psi}{d \alpha_j} = 0,
\end{equation*} 
and solving for $\alpha_j$ and then use above
equation to fiind $\alpha_k$.

Some care needs to be taken to satisfy the constraints on $\alpha_i$ during this 
optimization.
\es

\bs{SMO}
\vspace{.2in}
\fframe{
{\bf let} $D = \{(\ol{x}_1,y_1), (\ol{x}_2,y_2),\dots,(\ol{x}_l,y_l)\} \subset \Rnspace{n} \times \{+1,-1\}$\\
$\ol{\alpha} \leftarrow \ol{0}$\\
{\bf repeat}\\
\begin{minipage}{8in}\rm
\begin{enumerate}
\item pick two points, $\ol{x}_j$ and $\ol{x}_k$  in $D$ 
together with their\\
respective Lagrangian multipliers, $\alpha_j$ and $\alpha_k$, where $j \ne k$.
\item optimize the subproblem $\max \phi'(\alpha_j,\alpha_k)$ (keeping the other Lagrangian\\
multipliers constant).
\item compute $b$
\end{enumerate}
\end{minipage}
{\bf until} {\rm the KKT conditions hold for all $d\in D$}\\
{\bf return} $(\ol{\alpha},b)$
}
\begin{center}
Sequential Minimal Optimization.
\end{center}
\es

\end{document}
%%%%%%%%%%%%%%%%%%%%%%%%%%% end of template1.tex %%%%%%%%%%%%%%%%%%%%%%%%%%%%%%%%

