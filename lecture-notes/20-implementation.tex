%template1.tex
%The following LaTeX source file represents the simplest kind of slide presentation; no overlays, no included graphics. Substitute your favorite style for ``pascal''. To create the PDF file template1.pdf, (1) be sure to use the prosper class, then (2) execute the command latex template1.tex, and (3) the command dvipdf template1.dvi.

%%%%%%%%%%%%%%%%%%%%%%%%%%%%%%% template1.tex %%%%%%%%%%%%%%%%%%%%%%%%%%%%%%%%%%%
\documentclass[a4paper,blends,pdf,colorBG,slideColor]{prosper}
% definitions for slides for CSC544
% Lutz Hamel, (c) 2007

\hypersetup{pdfpagemode=FullScreen}

\usepackage{times}
\usepackage{latexsym}
\usepackage{alltt}
\usepackage{booktabs}
\usepackage{amsmath}
\usepackage{amsopn}
\usepackage{amsfonts}
\usepackage{amssymb}
%\usepackage[usenames]{color}

\def\sign{\qopname\relax{no}{sign}}
\def\argmax{\qopname\relax{no}{argmax}}
\def\argmin{\qopname\relax{no}{argmin}}

\newcommand{\grad}{\ensuremath{\nabla}} 
\newcommand{\loss}{\ensuremath{{\cal L}}}
\newcommand{\err}{\mbox{err}}
\newcommand{\mse}{\mbox{mse}}
\newcommand{\acc}{\mbox{acc}}
\newcommand{\Integer}{\ensuremath{\mathbb{N}}}
\newcommand{\size}[1]{{|{#1}|}}
\newcommand{\Rnspace}[1]{\ensuremath{\mathbb{R}^{#1}}}
\newcommand{\Real}{\ensuremath{\mathbb{R}}}
\newcommand{\mytt}[1]{{\small\tt{#1}}}
\newcommand{\textemph}[1]{{\em #1}}
\newcommand{\suchthat}{\mid}
\newcommand{\orbar}{\;|\;}
\newcommand{\bs}[1]{\begin{slide}{#1}\ptsize{8}}
\newcommand{\es}{\end{slide}}
\newcommand{\co}{\,\colon\;}
\newcommand{\pair}[2]{\ensuremath{( {#1}, {#2} )}}
\newcommand{\model}[1]{\hat{#1}}
\newcommand{\ul}[1]{{\bf\em #1}}
\newcommand{\ol}{\overline}
\newcommand{\definition}[1]{{\bf Definition: }{\em #1}}
\newcommand{\example}[1]{{\bf Example: }{#1}}
\newcommand{\abs}[1]{|{#1}|}
\newcommand{\mytab}{\makebox[.1in]{}}

\newcommand{\fdef}[1]{
\begin{center}
\fbox{
\begin{minipage}{3.5in}
{\bf Definition:}
{#1}
\end{minipage}
}
\end{center}
}

\newcommand{\fframe}[1]{
\begin{center}
\fbox{
\begin{minipage}{3.5in}
{#1}
\end{minipage}
}
\end{center}
}

\newcommand{\nframe}[1]{
\begin{center}
\begin{minipage}{3.5in}
{#1}
\end{minipage}
\end{center}
}

\newenvironment{Rcode}
	{
		\scriptsize
		\begin{quote}
		\begin{alltt}
	}
	{
		\end{alltt}
		\end{quote}
	}




\begin{document}

\bs{Gradient Ascent}
\small
Recall the following setting for training support vector machines.

Assume that we are given the training set,
\begin{equation*}
D = \{(\ol{x}_1,y_1),(\ol{x}_2,y_2),\ldots,(\ol{x}_l,y_l)\} \subseteq \Rnspace{n}\times\{+1, -1\}.
\end{equation*}
We are interested in computing a classifier in the form of a support vector machine model,
\begin{equation*}
\model{f}(\ol{x}) = \sign \left ( \sum_{i=1}^l y_i \alpha^*_i k(\ol{x}_i,\ol{x}) - b^*\right ),
\end{equation*}
using a training algorithm based on the Lagrangian dual,
\begin{equation*}
\ol{\alpha}^* = {\color{red} \argmax_{\ol{\alpha}} \phi'(\ol{\alpha})} = 
 \argmax_{\ol{\alpha}}  \left (\sum_{i=1}^l \alpha_i - 
  \frac{1}{2}\sum_{i=1}^l\sum_{j=1}^l y_i y_j \alpha_i \alpha_j k(\ol{x}_i,\ol{x}_j) \right ),
\end{equation*}
subject to  the constraints,
\begin{align*}
\sum_{i=1}^l y_i \alpha_i &= 0, \\
C \ge \alpha_i &\ge 0,  
\end{align*}
with $i=1,\ldots,l$.
\es

\bs{Gradient Ascent}
\small
In order to train SVMs we need to solve the optimization problem
\begin{equation*}
\ol{\alpha}^* = \argmax_{\ol{\alpha}} \phi'(\ol{\alpha}).
\end{equation*}
Perhaps the most straightforward implementation of  the Lagrangian dual optimization problem 
is by {\em gradient ascent}.

Formally, let $h$ be a differentiable function with respect to $\ol{x} \in \Rnspace{n}$, then the
\textemph{gradient} of $h$ is defined as,
\begin{equation*}
\grad h = \left (\frac{\partial h}{\partial x_1},\cdots,\frac{\partial h}{\partial x_n}\right ).
\end{equation*}
We often write, 
\begin{equation*}
\grad_{\! i} h = \frac{\partial h}{\partial x_i},
\end{equation*}
for the $i^\text{th}$ component of $\grad h$ with $i = 1,\ldots, n$.

Now, $\grad h(\ol{y})$ with $\ol{y}\in \Rnspace{n}$ is a vector that points in the direction of the largest
increase of $h$ at point $\ol{y}$.

We can use this to find the maximum of our dual $\phi'$ by simply following the gradient until the 
gradient becomes zero $\Rightarrow$ gradient ascent.
\es

\bs{Gradient Ascent}
\vspace{.2in}
\fframe{
{\bf let} $\eta \in[ 0,1]$\\
$\ol{\alpha}\leftarrow \ol{0}$\\
{\bf repeat}\\
\mytab $\ol{\alpha}_{\text{old}} \leftarrow \ol{\alpha}$\\
\mytab {\bf for} $i = 1 \text{ \bf to } l$ {\bf do}\\
\mytab\mytab $\alpha_i \leftarrow \alpha_i + \eta \grad_{\! i} \phi'(\ol{\alpha})$\\
\mytab {\bf end for}\\
{\bf until} $\ol{\alpha} - \ol{\alpha}_{\text{old}} \approx \ol{0}$\\
{\bf return} $\ol{\alpha}$
}
\begin{center}
The gradient ascent algorithm.
\end{center}
\es

\bs{Gradient Ascent}
{\bf Observation:} We have treated our optimization problem as an unconstrained optimization
problem ignoring the constraints,
\begin{align*}
\sum_{i=1}^l y_i \alpha_i &= 0, \\
C \ge \alpha_i &\ge 0,  
\end{align*}
with $i=1,\ldots,l$.
The first constraint is due to the optimization of the offset term $b$ and the second constraint
is the soft-margin constraint for the Lagrangian multipliers.

We can dispense with the first constraint by simply setting $b=0$.

The second constraint is easily implemented as a set of {\em box constraints} on $\ol{\alpha}$,
\begin{equation*}
\alpha_i \leftarrow \min\left\{C,\max\left\{0,\alpha_i + \eta\grad_{\! i} \phi'(\ol{\alpha})\right\}\right\}.
\end{equation*}

\es

\bs{The Kernel-Adatron}
\vspace{.2in}
\fframe{
{\bf let} $D = \{(\ol{x}_1,y_1), (\ol{x}_2,y_2),\dots,(\ol{x}_l,y_l)\} \subset \Rnspace{n} \times \{+1,-1\}$\\
{\bf let} $\eta > 0$\\
{\bf let} $C > 0$\\
{\bf let} $b = 0$\\
$\ol{\alpha} \leftarrow \ol{0}$\\
{\bf repeat}\\
\mytab $\ol{\alpha}_{\text{\rm old}} \leftarrow \ol{\alpha}$\\
\mytab {\bf for} $i = 1$ {\bf to} $l$ {\bf do}\\
\mytab\mytab$ \alpha_i \leftarrow \min\left \{C,\max\left\{0,\alpha_i + \eta-\eta y_i \sum_{j=1}^l y_j \alpha_j k(\ol{x}_j,\ol{x}_i)\right\}\right\}$ \\
\mytab {\bf end for}\\
{\bf until} $\ol{\alpha} - \ol{\alpha}_{\text{\rm old}} \approx \ol{0}$\\
{\bf return} $(\ol{\alpha},b)$
}
\begin{center}
The Kernel-Adatron algorithm.
\end{center}
\es

\end{document}
%%%%%%%%%%%%%%%%%%%%%%%%%%% end of template1.tex %%%%%%%%%%%%%%%%%%%%%%%%%%%%%%%%

