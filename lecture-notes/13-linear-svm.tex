%template1.tex
%The following LaTeX source file represents the simplest kind of slide presentation; no overlays, no included graphics. Substitute your favorite style for ``pascal''. To create the PDF file template1.pdf, (1) be sure to use the prosper class, then (2) execute the command latex template1.tex, and (3) the command dvipdf template1.dvi.

%%%%%%%%%%%%%%%%%%%%%%%%%%%%%%% template1.tex %%%%%%%%%%%%%%%%%%%%%%%%%%%%%%%%%%%
\documentclass[a4paper,blends,pdf,colorBG,slideColor]{prosper}
% definitions for slides for CSC544
% Lutz Hamel, (c) 2007

\hypersetup{pdfpagemode=FullScreen}

\usepackage{times}
\usepackage{latexsym}
\usepackage{alltt}
\usepackage{booktabs}
\usepackage{amsmath}
\usepackage{amsopn}
\usepackage{amsfonts}
\usepackage{amssymb}
%\usepackage[usenames]{color}

\def\sign{\qopname\relax{no}{sign}}
\def\argmax{\qopname\relax{no}{argmax}}
\def\argmin{\qopname\relax{no}{argmin}}

\newcommand{\grad}{\ensuremath{\nabla}} 
\newcommand{\loss}{\ensuremath{{\cal L}}}
\newcommand{\err}{\mbox{err}}
\newcommand{\mse}{\mbox{mse}}
\newcommand{\acc}{\mbox{acc}}
\newcommand{\Integer}{\ensuremath{\mathbb{N}}}
\newcommand{\size}[1]{{|{#1}|}}
\newcommand{\Rnspace}[1]{\ensuremath{\mathbb{R}^{#1}}}
\newcommand{\Real}{\ensuremath{\mathbb{R}}}
\newcommand{\mytt}[1]{{\small\tt{#1}}}
\newcommand{\textemph}[1]{{\em #1}}
\newcommand{\suchthat}{\mid}
\newcommand{\orbar}{\;|\;}
\newcommand{\bs}[1]{\begin{slide}{#1}\ptsize{8}}
\newcommand{\es}{\end{slide}}
\newcommand{\co}{\,\colon\;}
\newcommand{\pair}[2]{\ensuremath{( {#1}, {#2} )}}
\newcommand{\model}[1]{\hat{#1}}
\newcommand{\ul}[1]{{\bf\em #1}}
\newcommand{\ol}{\overline}
\newcommand{\definition}[1]{{\bf Definition: }{\em #1}}
\newcommand{\example}[1]{{\bf Example: }{#1}}
\newcommand{\abs}[1]{|{#1}|}
\newcommand{\mytab}{\makebox[.1in]{}}

\newcommand{\fdef}[1]{
\begin{center}
\fbox{
\begin{minipage}{3.5in}
{\bf Definition:}
{#1}
\end{minipage}
}
\end{center}
}

\newcommand{\fframe}[1]{
\begin{center}
\fbox{
\begin{minipage}{3.5in}
{#1}
\end{minipage}
}
\end{center}
}

\newcommand{\nframe}[1]{
\begin{center}
\begin{minipage}{3.5in}
{#1}
\end{minipage}
\end{center}
}

\newenvironment{Rcode}
	{
		\scriptsize
		\begin{quote}
		\begin{alltt}
	}
	{
		\end{alltt}
		\end{quote}
	}




\begin{document}

\bs{\large Dual Maximum Margin Optimization}
\fframe{{\bf Proposition:}
(The Maximum Margin Lagrangian Dual)
Given the primal maximum margin optimization, \footnote{See lecture notes on maximum margin
classifiers.}
then the Lagrangian dual optimization for maximum margin classifiers is
\begin{equation*}
\max_{\ol{\alpha}} \phi'(\ol{\alpha}) = 
  \max_{\ol{\alpha}} \left ( \sum_{i=1}^l \alpha_i - 
  \frac{1}{2}\sum_{i=1}^l \sum_{j=1}^l \alpha_i \alpha_j y_i y_j \ol{x}_i \bullet \ol{x}_j \right ),
\end{equation*}
subject to the constraints
\begin{align*}
\sum_{i=1}^l \alpha_i  y_i &= 0,\\
\alpha_i  &\ge 0, 
\end{align*}
with $i = 1,\ldots,l$.
}
\es


\bs{The Dual Decision Function}
We also know that given the optimal Lagrangian multipliers $\ol{\alpha}^*$ we can
construct both $\ol{w}^*$,
\begin{equation*}
\ol{w}^* = \sum_{i=1}^l \alpha^*_i y_i  \ol{x}_i,
\end{equation*}
and $b^*$,
\begin{equation*}
b^*  = \sum_{i=1}^l \alpha^*_i y_i  \ol{x}_i\bullet\ol{x}_{sv^+} - 1,
\end{equation*}
where we pick one support vector from the set of available support vectors,
\begin{equation*}
(\ol{x}_{sv^+},+1) \in \{(\ol{x}_i,+1) \mid (\ol{x}_i,+1)\in D \text{ and } \alpha^*_i > 0\},
\end{equation*}
The identity for $b^*$ follows directly from the KKT complimentarity condition with $\alpha^*_j > 0$ for some point $(\ol{x}_j,y_j)\in D$, then 
$y_j(\ol{w}^*\bullet\ol{x}_j-b^*) - 1 = 0$ or,
\begin{align*}
\ol{w}^*\bullet\ol{x}_j & = b^* + 1 \quad\text{ if $y_j = +1$},\\
\ol{w}^*\bullet\ol{x}_j & = b^* - 1 \quad\text{ if $y_j = -1$}.
\end{align*}
Plugging in $\ol{w}^*$ and solving for $b^*$ gives us our required result.


\es

\bs{The Dual Decision Function}
Putting this all together gives us,
\begin{align*}
\model{f}(\ol{x}) &= \sign (\ol{w}^*\bullet\ol{x} -b^*)\\
&= \sign \left ( \sum_{i=1}^l \alpha^*_i y_i \ol{x}_i\bullet\ol{x} - \sum_{i=1}^l \alpha^*_i y_i  \ol{x}_i\bullet\ol{x}_{sv^+} + 1\right ).
\end{align*}
As we would expect, the dual decision function is completely determined by the
Lagrangian multipliers $\ol{\alpha}^*$.

Observing that the decision function is completely determined by points $\ol{x}_i$ with
$\alpha^*_i > 0$, we can say that the dual decision function is completely determined by
the support vectors.

\begin{center}
\fbox{We consider the dual decision function a {\em support vector machine}.}
\end{center}

\es

\bs{Linear SVMs}
Given
\begin{itemize}
\item a dot product space $\Rnspace{n}$ as our data universe with points $\ol{x} \in \Rnspace{n}$,
\item some target function $f\co \Rnspace{n} \rightarrow \{+1,-1\}$,
\item a labeled, linearly separable training set, 
\begin{equation*}
D = \{(\ol{x}_1,y_1),(\ol{x}_2,y_2),\ldots,(\ol{x}_l,y_l)\} \subseteq \Rnspace{n}\times\{+1, -1\},
\end{equation*}
where $y_i = f(\ol{x}_i)$,
\end{itemize}
then compute a model  $\model{f}\co \Rnspace{n} \rightarrow \{+1,-1\}$ using $D$ such that,
\begin{equation*}
\model{f}(\ol{x}) \cong f(\ol{x}),
\end{equation*}
for all $\ol{x} \in \Rnspace{n}$.

\es

\bs{Linear SVMs}
Here we take as our models the linear support vector machines,
\begin{equation*}
\model{f}(\ol{x}) = \sign \left ( \sum_{i=1}^l \alpha^*_i y_i \ol{x}_i\bullet\ol{x} - \sum_{i=1}^l \alpha^*_i y_i  \ol{x}_i\bullet\ol{x}_{sv^+} + 1\right ),
\end{equation*}
and we compute our support vector models with the Lagrangian dual optimization for maximum margin classifiers,
\begin{equation*}
\ol{\alpha}^* =  
  \argmax_{\ol{\alpha}} \left ( \sum_{i=1}^l \alpha_i - 
  \frac{1}{2}\sum_{i=1}^l \sum_{j=1}^l \alpha_i \alpha_j y_i y_j \ol{x}_i \bullet \ol{x}_j \right ),
\end{equation*}
subject to the constraints
\begin{align*}
\sum_{i=1}^l \alpha_i  y_i &= 0,\\
\alpha_i  &\ge 0, 
\end{align*}
where $i = 1,\ldots,l$.
\es

\end{document}
%%%%%%%%%%%%%%%%%%%%%%%%%%% end of template1.tex %%%%%%%%%%%%%%%%%%%%%%%%%%%%%%%%

