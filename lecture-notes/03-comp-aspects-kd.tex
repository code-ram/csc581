%template1.tex
%The following LaTeX source file represents the simplest kind of slide presentation; no overlays, no included graphics. Substitute your favorite style for ``pascal''. To create the PDF file template1.pdf, (1) be sure to use the prosper class, then (2) execute the command latex template1.tex, and (3) the command dvipdf template1.dvi.

%%%%%%%%%%%%%%%%%%%%%%%%%%%%%%% template1.tex %%%%%%%%%%%%%%%%%%%%%%%%%%%%%%%%%%%
\documentclass[a4paper,blends,pdf,colorBG,slideColor]{prosper}
% definitions for slides for CSC544
% Lutz Hamel, (c) 2007

\hypersetup{pdfpagemode=FullScreen}

\usepackage{times}
\usepackage{latexsym}
\usepackage{alltt}
\usepackage{booktabs}
\usepackage{amsmath}
\usepackage{amsopn}
\usepackage{amsfonts}
\usepackage{amssymb}
%\usepackage[usenames]{color}

\def\sign{\qopname\relax{no}{sign}}
\def\argmax{\qopname\relax{no}{argmax}}
\def\argmin{\qopname\relax{no}{argmin}}

\newcommand{\grad}{\ensuremath{\nabla}} 
\newcommand{\loss}{\ensuremath{{\cal L}}}
\newcommand{\err}{\mbox{err}}
\newcommand{\mse}{\mbox{mse}}
\newcommand{\acc}{\mbox{acc}}
\newcommand{\Integer}{\ensuremath{\mathbb{N}}}
\newcommand{\size}[1]{{|{#1}|}}
\newcommand{\Rnspace}[1]{\ensuremath{\mathbb{R}^{#1}}}
\newcommand{\Real}{\ensuremath{\mathbb{R}}}
\newcommand{\mytt}[1]{{\small\tt{#1}}}
\newcommand{\textemph}[1]{{\em #1}}
\newcommand{\suchthat}{\mid}
\newcommand{\orbar}{\;|\;}
\newcommand{\bs}[1]{\begin{slide}{#1}\ptsize{8}}
\newcommand{\es}{\end{slide}}
\newcommand{\co}{\,\colon\;}
\newcommand{\pair}[2]{\ensuremath{( {#1}, {#2} )}}
\newcommand{\model}[1]{\hat{#1}}
\newcommand{\ul}[1]{{\bf\em #1}}
\newcommand{\ol}{\overline}
\newcommand{\definition}[1]{{\bf Definition: }{\em #1}}
\newcommand{\example}[1]{{\bf Example: }{#1}}
\newcommand{\abs}[1]{|{#1}|}
\newcommand{\mytab}{\makebox[.1in]{}}

\newcommand{\fdef}[1]{
\begin{center}
\fbox{
\begin{minipage}{3.5in}
{\bf Definition:}
{#1}
\end{minipage}
}
\end{center}
}

\newcommand{\fframe}[1]{
\begin{center}
\fbox{
\begin{minipage}{3.5in}
{#1}
\end{minipage}
}
\end{center}
}

\newcommand{\nframe}[1]{
\begin{center}
\begin{minipage}{3.5in}
{#1}
\end{minipage}
\end{center}
}

\newenvironment{Rcode}
	{
		\scriptsize
		\begin{quote}
		\begin{alltt}
	}
	{
		\end{alltt}
		\end{quote}
	}




\begin{document}

\bs{Computational Aspects of KD}
\begin{itemize}
\item Data Access 
	\begin{itemize}
	\item read.csv
	\item write.csv
	\item edit
	\end{itemize}
\item Visualization 
	\begin{itemize}
	\item scatter plots
	\end{itemize}
\item Data Manipulation
	\begin{itemize}
	\item attribute-oriented approach
	\item observation-oriented approach
	\end{itemize}
\item Model Building and Evaluation
\item Model Deployment
\end{itemize}
\es


\bs{Data Manipulation}
Recall that a data frame is a data table
representation in R,
\begin{Rcode}
> mammals.df
  Legs Wings  Fur Feathers Mammal
1    4    no  yes       no   true
2    2   yes   no      yes  false
3    4    no   no       no  false
4    4   yes  yes       no   true
5    3    no   no       no  false
\end{Rcode}
\es

\bs{Data Manipulation}
{\bf Attribute-oriented Approach}

We can access any attribute in the mammals data frame with the \mytt{\$} notation.
\begin{Rcode}
> mammals.df$Legs
[1] 4 2 4 4 3
> mammals.df$Mammal
[1]  true   false  false  true   false
Levels:  false  true
\end{Rcode}

R allows us to select groups of attributes with the \mytt{subset} function,
\begin{Rcode}
> subset(mammals.df, select=Fur:Mammal)
   Fur Feathers Mammal
1  yes       no   true
2   no      yes  false
3   no       no  false
4  yes       no   true
5   no       no  false
> subset(mammals.df, select=-Mammal)
  Legs Wings  Fur Feathers
1    4    no  yes       no
2    2   yes   no      yes
3    4    no   no       no
4    4   yes  yes       no
5    3    no   no       no
\end{Rcode}
\es

\bs{Data Manipulation}
{\bf Observation-oriented Approach}

We can use the \emph{subset} function also for observation-oriented data manipulation.

\begin{Rcode}
> subset(mammals.df, Legs == 4)
  Legs Wings  Fur Feathers Mammal
1    4    no  yes       no   true
3    4    no   no       no  false
4    4   yes  yes       no   true
\end{Rcode}

Another, slightly more complicated example,

\begin{Rcode}
> mammal.levels <- levels(mammals.df$Mammal)
> mammal.levels
[1] "false" "true" 
> true.level <- mammal.levels[2]
> subset(mammals.df, Mammal == true.level)
  Legs Wings Fur Feathers Mammal
1    4    no yes       no   true
4    4   yes yes       no   true
\end{Rcode}
\es


\bs{Model Building and Evaluation}
We use the library 'e1071' (don't ask :) for building support vector machine models. \footnote{The library 
is available through the Package Installer.}

\begin{Rcode}
> library(e1071)
\end{Rcode}
Now we can construct a support vector machine model of our mammals data with the \mytt{svm}
function,
\begin{Rcode}
> model<-svm(Mammal ~ .,data=mammals.df,kernel="linear")
\end{Rcode}

At this point we can evaluate our model by checking how it performs on the training set.

\begin{Rcode}
> mammals.df$Mammal == fitted(model)
[1] TRUE TRUE TRUE TRUE TRUE
\end{Rcode}

\vspace{1in}
\es


\bs{Model Deployment}
Model deployment means applying your model in an appropriate context.  In R we use
the \mytt{predict} function to compute the value of the dependent attribute for some object.
Given that R is a programming language we could program appropriate functionality around
the predict function.

\begin{Rcode}
> independent.df <- subset(mammals.df, select=-Mammal)
> predict(model, independent.df)
     1      2      3      4      5 
  true  false  false   true  false 
Levels:  false  true
\end{Rcode}

How could we test in R whether these predictions are correct with respect to the original
data set \mytt{mammals.df}?
\es

\end{document}
%%%%%%%%%%%%%%%%%%%%%%%%%%% end of template1.tex %%%%%%%%%%%%%%%%%%%%%%%%%%%%%%%%

